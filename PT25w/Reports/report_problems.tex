\documentclass[11pt]{article}
\usepackage[utf8]{inputenc}
\usepackage[T1]{fontenc}
\usepackage{CJKutf8}
\usepackage{geometry}
\usepackage{amsmath,amssymb}
\usepackage{enumitem}
\geometry{margin=1in}

\title{レポート課題まとめ(Classes 4--7)}
\date{}

\begin{document}
\begin{CJK}{UTF8}{min}
\maketitle

\section*{Class 4: Kolmogorovの拡張定理}
\begin{enumerate}[label=\textbf{4.\arabic*}]
  \item \(\mu\) が well-defined であることを確かめよ。具体的には \(\pi_n^{-1}(A_n)=\pi_m^{-1}(A_m)\) となる \(A_n,A_m\in \mathcal{S}\) について,\(P_n(A_n)=P_m(A_m)\) が成り立つことを示せ。
  \item 授業の証明では \(\Omega_1 = \{-1,1\}\) における一様確率測度の無限直積を扱った。全く同じ証明は \([0,1]\) 上の一様測度の無限測度の構成では成り立たない?なぜか述べよ(拡張するにはより複雑な議論が必要になる理由を説明せよ)。
\end{enumerate}

\section*{Class 5: 様々な分布}
\begin{enumerate}[label=\textbf{5.\arabic*}]
  \item \textbf{ポアソン分布に関する問題}:\quad \(X\sim\mathrm{Poi}(\lambda)\) の期待値と分散を求めよ。また,\(X_1\sim\mathrm{Poi}(\lambda_1)\) と \(X_2\sim\mathrm{Poi}(\lambda_2)\) が独立のとき,\(X_1+X_2\) の分布を求めよ。
  \item \textbf{幾何分布に関する問題}:\quad \(X\sim\mathrm{Geo}(p)\)(\(\mathbb P(X=k)=(1-p)^{k-1}p\))の期待値と分散を求めよ。さらに記憶なし性質
  \[
    \mathbb{P}(X>m+n\mid X>m)=\mathbb{P}(X>n)
  \]
  が成り立つことを示せ。
  \item ガンマ分布がどのような応用を持つか調べて記述せよ。
\end{enumerate}

\section*{Class 6: 大数の法則と中心極限定理(コイントス)}
\begin{enumerate}[label=\textbf{6.\arabic*}]
  \item \textbf{理論計算(一般 \(p\))}:\quad \(X_i\sim \mathrm{CF}(p)\)(すなわち \(\mathbb P(X_i=1)=p\), \(\mathbb P(X_i=0)=1-p\))とし,\(S_n\) の期待値 \(\mathbb E[S_n]\) と分散 \(\operatorname{Var}(S_n)\) を求めよ。
  \item \textbf{シミュレーション(生成AIを用いること)}:
  \begin{itemize}[label=--,leftmargin=1.4em]
    \item \(n=100\) としてコイントス 100 回を 10000 回繰り返すシミュレーションを実装せよ。
    \item 得られた \(S_n\) のヒストグラムを作成せよ。
    \item 同じグラフ上に(\(p=\tfrac12\) のとき)平均 \(\tfrac{n}{2}\)・分散 \(\tfrac{n}{4}\) の正規分布密度を重ねて描け(必要なら連続性補正も可視化)。
  \end{itemize}
  \item \textbf{考察}:
  \begin{itemize}[label=--,leftmargin=1.4em]
    \item シミュレーション結果と正規近似の一致度を,中心と裾の両方で評価せよ。
    \item \(n=20, 50, 200\) などに変えたときの精度の違いを述べよ。
    \item 連続性補正の効果について論じよ。
  \end{itemize}
\end{enumerate}

\section*{Class 7: 特性関数}
\begin{enumerate}[label=\textbf{7.\arabic*}]
  \item \textbf{幾何分布とガンマ分布の特性関数}:\quad 次の分布について特性関数 \(\varphi_X(\xi)=\mathbb{E}[e^{i\xi X}]\) を計算せよ(わからなかったら導出法を調べて書け)。
  \begin{itemize}[label=--,leftmargin=1.4em]
    \item 成功確率 \(p\in(0,1)\) の幾何分布(試行回数型)。\(\mathbb{P}(X=k)=p(1-p)^{k-1}\)(\(k=1,2,\dots\))。
    \item 形状 \(k>0\),スケール \(\theta>0\) のガンマ分布。密度 \(f(x)=\dfrac{1}{\Gamma(k)\theta^k}x^{k-1}e^{-x/\theta}\)(\(x>0\))。
  \end{itemize}
  \item \textbf{弱法則の向きから見る特性関数の収束}:\quad 独立同分布な確率変数 \(X_i\)(\(i\ge 1\))が \(\mathbb{E}[X_i]=0\) かつ \(\mathbb{E}[X_i^2]<\infty\) を満たすとする。\(S_n := \tfrac{1}{n}\sum_{i=1}^n X_i\) とおくと,\(\varphi_{S_n}(\xi)\to 1\)(各 \(\xi\) に対して)を示せ(授業の定理より \(S_n\) が 0 に分布収束している事実を使える)。
  \item \textbf{ヒント}:\quad テイラーの定理を用いて \(e^{i\xi X_i/n}=1+\tfrac{i\xi X_i}{n}+\bigl(\tfrac{i\xi \theta X_i}{n}\bigr)^2\)(\(\theta\in[0,1]\))と書き,期待値を取って評価する。有限分散条件から二乗項は \(n^{-2}\) オーダーで消えることを確認せよ。
\end{enumerate}

\end{CJK}
\end{document}
