\documentclass[uplatex]{bxjsarticle}
\usepackage{amsmath,amssymb}
\usepackage{enumitem}

\title{レポート課題集}
\author{}
\date{}

\begin{document}
\maketitle

\section*{クラス1}
講義内容を踏まえ, 以下の問いについてレポートとしてまとめてください。必要に応じて参考文献や図表を添付し, 根拠を明確に示しましょう。
\begin{enumerate}[label=\arabic*.]
\item 歴史の中で興味を持った出来事または人物を一つ選び, 当時の背景や影響を深掘りせよ。
\item $[0, 1]$ 内のすべての部分集合に確率が定義できると仮定したときに生じる問題点を議論せよ。
\item 単調増加列に対する連続性を用いて、可算劣加法性を示せ: $$P\bigl(\bigcup_{i=1}^{\infty} A_i\bigr) \le \sum_{i=1}^{\infty} P(A_i)$$
\end{enumerate}

\section*{クラス2}
以下の問いについてレポートとしてまとめよ。(例えば、関連するより簡単な問題を証明するや、シミュレーションで確かめるなど、また応用例を調べるなどでも良い)必要に応じて参考文献や図表を添付し, 根拠を明確に示せ。
\begin{enumerate}[label=\arabic*.]
\item Feigeの予想(期待値$1$である独立な非負値確率変数の列$(X_i)$に対し, \mathbb{P}\Big(\sum^n_{i=1} X_i \leq n+1\Big)\geq 1/e$)に関し、考察せよ。
\end{enumerate}

\section*{クラス3}
次を確かめよ
\begin{enumerate}[label=\arabic*.]
\item サイコロを3回ふる。すべての目を足し合わせた確率変数の期待値を求めよ
\item 有限確率空間での期待値の定義と一般の期待値の定義が一致することを確かめよ。このとき、$\mathcal F=2^\Omega$ととることに注意
\item $X$を単関数とし, (可測)写像 $g\colon \mathbb{R}\to[0,\infty)$が$\mathbb{E}[g(X)]<\infty$を満たすとする。これに対し $\displaystyle \mathbb{E}[g(X)]=\int_{\mathbb{R}} g(x)\,d\mu_X$ を確かめよ。
\end{enumerate}

\end{document}
